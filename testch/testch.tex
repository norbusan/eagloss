%!TEX TS-program = xelatex
%!TEX encoding = UTF-8 Unicode
\documentclass{article}

\usepackage{hyperref} 


% Bibliography
\usepackage{natbib}


% Fonts
\usepackage{fontspec} 
\usepackage{xeCJK} 
\setmainfont[Mapping=tex-text]{Times New Roman} % rm
\setsansfont[Mapping=tex-text]{Arial}           % sf
\setmonofont{Courier New}                       % tt
\setCJKmainfont{HAN NOM A} %xelatex 標楷體
%\setCJKmonofont{MingLiU}  %xelatex 細明體
\xeCJKDeclareSubCJKBlock{ExtB}{ "20000 -> "2A6DF }
\setCJKmainfont[ExtB=HAN NOM B,AutoFakeBold=1.5]{HAN NOM A}
\setCJKsansfont[ExtB=HAN NOM B,AutoFakeBold=1.5]{HAN NOM A}
%\newfontfamily\greekfont[Mapping=tex-text]{Times New Roman}

% Quotation marks
\newif\ifbritquote\britquotefalse  
%\britquotetrue % uncomment for British
\newif\ifinsidequote\insidequotefalse
\def\qt#1{\ifinsidequote\insidequotefalse\insidequote{#1}\insidequotetrue\else\insidequotetrue\outsidequote{#1}\insidequotefalse\fi}
\def\insidequote#1{\ifbritquote ``#1"\else `#1'\fi}
\def\outsidequote#1{\ifbritquote `#1'\else ``#1"\fi}
% and for Chinese
\def\cqt#1{\ifinsidequote\insidequotefalse‘#1’\insidequotetrue\else\insidequotetrue“#1”\insidequotefalse\fi}
% also 书名号
\def\smh#1{《\hspace*{-2pt}#1\hspace*{-2pt}》}

% Footnotes
\usepackage[hang,flushmargin]{footmisc} 

\newif\iffootnote
\let\Footnote\footnote
\renewcommand\footnote[1]{\begingroup\footnotetrue\Footnote{#1}\endgroup}



% Index
\usepackage{makeidx}
\makeindex


% Database for glossary
\usepackage{longdata}  % this is an extension of datatools.sty that allows \par in fields

%commands to get a punctuation to avoid interaction with delimiters in the database (CSV)
\def\comma{,}
\def\semicolon{;}
\def\colon{:}  

% test for blanks
\newcommand\ifblankcs[1]{\edef\doit{\noexpand\ifblank{#1}}\doit}
  
%  Technical terms database and commands
\DTLloaddb{terms}{terms.csv}

\def\term#1#2{\DTLforeach*[\DTLiseq{\name}{#1}]{terms}{\hanzi=hanzi,\name=name,\py=pinyin,\gl=gloss,\nt=notone,\ot=other,\otnt=othernotone,\desc=description}{\def\notone{\index{\nt @\textit{\py}}\nt}\def\gloss{\index{\gl}\gl}\def\pinyin{\index{\nt @\textit{\py}}\py}\def\other{\index{\otnt @\textit{\ot}}\ot}#2}}

\def\tm#1{\term{#1}{\textit{\pinyin}}}  % primary command
\def\tmf#1{\term{#1}{\textit{\pinyin} \hanzi (\ifblankcs{\otnt}{}{{\it\other},\ }\gloss)}}	% command for first use
\def\tme#1{\term{#1}{\gloss}}							% English gloss only
\def\tmc#1{\term{#1}{\textit{\pinyin} \hanzi}}			% Chinese (pinyin and hanzi )

%  Books database and commands
\DTLloaddb{books}{books.csv}

\def\book#1#2{%
\DTLforeach*[\DTLiseq{\name}{#1}]{books}%
{\hanzi=hanzi,\name=name,\py=pinyin,\nt=notone,\gl=gloss,\ot=other,\otnt=othernotone,\desc=description}%
{\def\gloss{\index{\gl @\textit{\gl}}\gl}%
\def\notone{\index{\nt @\textit{\nt}}\nt}%
\def\pinyin{\index{\nt @\textit{\nt}}\py}%
\def\other{\ifblankcs{\otnt}{}{\index{\otnt @\textit{\ot}}\ot}}%
#2}}

\def\bk#1{\book{#1}{\textit{\notone}}}                                % primary command
\def\bkf#1{\iffootnote\bkn{#1}\else\bk{#1}\footnote{\bkn{#1}}\fi}     % command for first use
\def\bkn#1{\book{#1}{\textit{\pinyin}\smh{\hanzi}\other\ (\gloss)}}	  % complete book title

%  People database and commands
\DTLloaddb{people}{people.csv}

\def\person#1#2{\DTLforeach*[\DTLiseq{\name}{#1}]{people}{\hanzi=hanzi,\name=name,\py=pinyin,\nt=notone,\ot=other,\otnt=othernotone,\desc=description}{\def\notone{\index{\nt}\nt}\def\pinyin{\index{\nt}\py}\def\other{\index{\otnt @{\ot}}\ot}#2}}

\def\pn#1{\person{#1}{\notone}}  % primary command
\def\pnf#1{\iffootnote\pnn{#1}\else\pn{#1}\footnote{\pnn{#1}}\fi}  % first use
\def\pno#1{\person{#1}{\other}}							% other name only
\def\pnn#1{\person{#1}{\pinyin \hanzi \ifblankcs{\otnt}{}{\ \other}}}	    % complete  name

% Abbreviations
\def\bce{B.C.E.}
\usepackage[super]{nth}   % for writing \nth{3} for 3rd etc

\begin{document}

Some debate as to whether \tmf{ming2} should be interpreted only as \qt{name} or also as \qt{concept}. There is much discussion of \tm{ming2} in early Chinese philosophy both by \tme{rujia}s, especially \pnf{xunzi}, and by the \tme{mojia}s. The most elaborate treatment of the topic is contained in the \bkf{mojing}. For more detail see \cite{dongzt1998}.


\bibliography{testch}
\bibliographystyle{apa-good}
\nocite{*}

\newpage
%%%%%%%%%  GLOSSARY OF PEOPLE 
\section*{Glossary of People}%

\DTLsort*{index}{people}
\DTLforeach*{people}%
{\hanzi=hanzi,\name=name,\pinyin=pinyin,\notone=notone,\desc=description,\other=other,\othernotone=othernotone,\wadegiles=wadegiles,\inglossary=inglossary}{%
\ifblankcs{\inglossary}{\noindent%
\textbf{\textit{\pinyin}}\index{\notone} \hanzi %
\ifblankcs{\othernotone}{\ifblankcs{\wadegiles}{}%
    {\wadegiles\index{\wadegiles}}}%
    {\other\index{\othernotone @{\other}}}%
, \desc.\\[4.5pt]}%
{}}

%%%%%%%%%  GLOSSARY OF TEXTS
\section*{Glossary of Texts}%
\DTLforeach*{books}%
{\hanzi=hanzi,\name=name,\pinyin=pinyin,\gloss=gloss,\notone=notone,\desc=description,\other=other,\othernotone=othernotone,\inglossary=inglossary}{%
\ifblankcs{\inglossary}{\noindent%
\textbf{\textit{\pinyin}}\index{\notone @\textit{\notone}}《\hspace*{-2pt}\hanzi \hspace*{-2pt}》(\ifblankcs{\othernotone}{}{\other\index{\othernotone @\textit{\other}}, }\gloss\index{\gloss @\textit{\gloss}}), \desc.\\[5pt]}%
{}}


%%%%%%%%%  GLOSSARY OF TECHNICAL TERMS
\section*{Glossary of Technical Terms}%
\DTLforeach*{terms}%
{\hanzi=hanzi,\name=name,\pinyin=pinyin,\gloss=gloss,\notone=notone,\desc=description,\other=other,\othernotone=othernotone,\inglossary=inglossary}{%
\ifblankcs{\inglossary}{\noindent%
\textbf{\textit{\pinyin}}\index{\notone @\textit{\pinyin}}\index{\gloss} \hanzi (\ifblankcs{\othernotone}{}{{\it \other}, \index{\othernotone @\textit{\other}}}\gloss), \desc.\\[5pt]}%
{}}


%%%%%%%%% INDEX
\printindex

\end{document}

